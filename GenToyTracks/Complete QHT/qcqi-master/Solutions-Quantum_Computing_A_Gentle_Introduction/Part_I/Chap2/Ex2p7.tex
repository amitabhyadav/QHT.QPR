% Exercise 2.7
\section{}

For each of the following states, describe all orthonormal bases that include that state.
\begin{enumerate}[label=\alph*.,ref={Ex.~\thesection\alph*}]
\label{ex:2p7}
    % a.
    \item $\spket{}{0}{+ \ibf}{1}$
    % b.
    \item $\tfrac{1 + \ibf}{2} \ket{0} - \tfrac{1 - \ibf}{2} \ket{1}$
    % c.
    \item $\spket{}{0}{+ e^{\ibf \pi / 6}}{1}$
    % d.
    \item $\tfrac{1}{2} \ket{+} - \tfrac{\ibf \sqrt{3}}{2} \ket{-}$
\end{enumerate}

{\Sol}
There are a number of methods with which we can construct an orthonormal basis that
includes a given state.

\begin{enumerate}[label=\alph*.,ref={Sol.~\thesection\alph*}]
\label{sol:2p7}
    % a.
    \item \label{sol:2p7a} This is the $\ket{\ibf}$ state, so
    \begin{align*}
        \left \{ e^{\ibf \phi} \iket, e^{\ibf \phi} \imket \right \} = e^{\ibf \phi} \left \{ \ket{\ibf}, \ket{-\ibf} \right \}
    \end{align*}
    for $\phi = \R$ describe all orthonormal bases including the given state.
    % b.
    \item Note that this is the same state as in \ref{sol:2p7a}, multiplied by a complex constant factor of modulus one, $\tfrac{1+\ibf}{\sqrt{2}}$.
    Therefore, the orthonormal bases including the given state are again,
    \begin{align*}
        e^{\ibf \phi} \left \{ \ket{\ibf}, \ket{-\ibf} \right \},
    \end{align*}
    where we have absorbed the constant factor into the global phase, $e^{\ibf \phi}$.
    % c.
    \item We will solve this in two ways.
    First, recall that an n-dimensional state vector, $\ket{v}$, in Hilbert space can be written in an arbitrary basis as,
    \begin{align*}
        \ket{v} = \begin{pmatrix} a_1 \\ . \\ . \\ . \\ a_n \end{pmatrix},
    \end{align*}
    and its complex conjugate as,
    \begin{align*}
        \bra{v} = \begin{pmatrix} \overline{a_1}, \ldots, \overline{a_n} \end{pmatrix}.
    \end{align*}
    To find an orthonormal state to our given state, we require,
    \begin{align*}
        \begin{pmatrix} 1 & e^{-\ibf \pi/6} \end{pmatrix} \begin{pmatrix} a_1 \\ a_2 \end{pmatrix} = 0 \\
        \Rightarrow a_1 = -e^{-\ibf \pi / 6},~a_2 = 1,
    \end{align*}
    giving an orthonormal basis state,
    \begin{align*}
        \ket{v^\perp} = \spket{-e^{-\ibf \pi/6}}{0}{+}{1}
    \end{align*}

    Another way to solve this problem is to recast the given state vector in the \textit{extended complex plane}, $\C \cup \infty$.
    The correspondence betweeen $\C$ and the given state is given by,
    \begin{align*}
        \spket{}{0}{+ e^{\ibf \pi / 6}}{1} \mapsto \tfrac{e^{\ibf \pi / 6}}{\sqrt{2}} / \tfrac{1}{\sqrt{2}} = e^{\ibf \pi / 6},
    \end{align*}
    and can be represented by the complex number $\alpha = \tfrac{\sqrt{3}}{2} + \tfrac{\ibf}{2}$.
    This corresponds to the point $(\tfrac{\sqrt{3}}{2}, \tfrac{\ibf}{2}, 0)$ on the Bloch sphere with antipodal point
    $(-\tfrac{\sqrt{3}}{2}, -\tfrac{\ibf}{2}, 0)$.
    Expressed back in the extended complex plane, the antipodal point is $\alpha = -e^{\ibf \pi/6}$, and using the inverse of the map
    above,
    \begin{align*}
        \alpha = -e^{\ibf \pi/6}  \mapsto & \spket{}{0}{-e^{\ibf \pi/6}}{1} \\
        &\sim \spket{-e^{-\ibf \pi/6}}{0}{+}{1} \\
        &= \ket{v^\perp}
    \end{align*}
    Note that $-e^{\ibf \pi/6} = e^{7 \ibf \pi/6}$, so, equivalently,
    \begin{align*}
        \ket{v^\perp} = \spket{}{0}{+ e^{7 \ibf \pi/6}}{1}.
    \end{align*}
    Therefore, the orthonormal bases containing the given vector are,
    \begin{align*}
         \left \{ \spket{}{0}{+ e^{\ibf \pi / 6}}{1}, \tfrac{1}{\sqrt{2}} (\ket{0} -  e^{\ibf \pi/6} \ket{1} ) \right \}.
    \end{align*}
    modulo a global phase.
    % d.
    \item \label{sol:2p7d} Recall the ``swap and negate'' trick from \ref{sol:2p2d}.
    Applying the same procedure here, we obtain,
    \begin{align*}
        \ket{v'} = \tfrac{\ibf \sqrt{3}}{2} \ket{+} + {\tfrac{1}{2}} \ket{-}.
    \end{align*}
    But wait,
    \begin{align*}
        \bra{v'}\ket{v} = -\tfrac{\ibf \sqrt{3}}{4} - \tfrac{\ibf \sqrt{3}}{4} = - \tfrac{\ibf \sqrt{3}}{2} \neq 0~(!)
    \end{align*}
    That is, the shortcut does not apply here because we are dealing with imaginary amplitudes.
    Let's try again, this time being a bit smarter about it:
    \begin{align*}
        \begin{pmatrix} \tfrac{1}{2} & \tfrac{\ibf \sqrt{3}}{2} \end{pmatrix} \begin{pmatrix} a_1 \\ a_2 \end{pmatrix} = 0 \\
        \Rightarrow a_1 = - \ibf \sqrt{3} a_2.
    \end{align*}
    Now, we can choose either $a_1 = - \ibf \sqrt{3}, a_2 = 1$ and get
    \begin{align*}
        \ket{v^\perp} = \tfrac{- \ibf \sqrt{3}}{2} \ket{+} + {\tfrac{1}{2}} \ket{-},
    \end{align*}
    or $a_1 = \ibf \sqrt{3}, a_2 = -1$ for
    \begin{align*}
        \ket{v^\perp} = \tfrac{\ibf \sqrt{3}}{2} \ket{+} - {\tfrac{1}{2}} \ket{-}
    \end{align*}
    as these both represent the same state, \textit{i.e.} they differ by global phase.
    Therefore, the orthonormal basis that includes the given state is,
    \begin{align*}
         \left \{ \tfrac{1}{2} \ket{+} - \tfrac{\ibf \sqrt{3}}{2} \ket{-}, \tfrac{- \ibf \sqrt{3}}{2} \ket{+} + {\tfrac{1}{2}} \ket{-} \right \}.
    \end{align*}
    modulo a global phase.
\end{enumerate}
