% Exercise 2.6
\section{}

For each pair consisting of a state and a measurement basis, describe the possible
measurement outcomes and give the probability for each outcome.
\begin{enumerate}[label=\alph*.,ref={Ex.~\thesection\alph*}]
\label{ex:2p6}
    % a.
    \item $\tfrac{\sqrt{3}}{2} \ket{0} - \tfrac{1}{2} \ket{1},~\left \{ \ket{0}, \ket{1} \right \}$
    % b.
    \item $\tfrac{\sqrt{3}}{2} \ket{1} - \tfrac{1}{2} \ket{0},~\left \{ \ket{0}, \ket{1} \right \}$
    % c.
    \item $\ket{- \ibf},~\left \{ \ket{0}, \ket{1} \right \}$
    % d.
    \item $\ket{0},~\left \{ \ket{+}, \ket{-} \right \}$
    % e.
    \item $\spket{}{0}{-}{1},~\left \{ \ket{\ibf}, \ket{- \ibf} \right \}$
    % f.
    \item $\ket{1},~\left \{ \ket{\ibf}, \ket{- \ibf} \right \}$
    % g.
    \item $\ket{+},~\left \{ \tfrac{1}{2} \ket{0} + \tfrac{\sqrt{3}}{2} \ket{1}, \tfrac{\sqrt{3}}{2} \ket{0} - \tfrac{1}{2} \ket{1} \right \}$
\end{enumerate}

{\Sol}
We will denote the states here as $\ket{v}$.

\begin{enumerate}[label=\alph*.,ref={Sol.~\thesection\alph*}]
\label{sol:2p6}
    % a.
    \item \label{sol:2p6a} Let's write this first one in some detail.
    Given the standard basis, the measured values can end up in either $\ket{0}$ or $\ket{1}$.
    The probability for state $\ket{0}$ is,
    \begin{align*}
        P(\ket{0}) &= \left | \bra{0} \cdot \left ( \tfrac{\sqrt{3}}{2} \ket{0} - \tfrac{1}{2} \ket{1} \right ) \right |^2 \\
        &= \left | \tfrac{\sqrt{3}}{2} \braket{0} \right |^2 \\
        &= \tfrac{3}{4}
    \end{align*}
    and state $\ket{1}$,
    \begin{align*}
        P(\ket{1}) &= \left | \bra{1} \cdot \left ( \tfrac{\sqrt{3}}{2} \ket{0} - \tfrac{1}{2} \ket{1} \right ) \right |^2 \\
        &= \left | -\tfrac{1}{2} \braket{1} \right |^2 \\
        &= \tfrac{1}{4}
    \end{align*}
    % b.
    \item This is the same as \ref{sol:2p6a} above, but with amplitudes swapped:
    $P(\ket{0}) = \tfrac{1}{4}$ and $P(\ket{1}) = \tfrac{3}{4}$.
    % c.
    \item Since $\ket{-\ibf} = \tfrac{1}{\sqrt{2}} (\ket{0} - \ibf \ket{1})$, $P(\ket{0}) = P(\ket{1}) = \tfrac{1}{2}$.
    % d.
    \item $P(\ket{+}) = P(\ket{-}) = \tfrac{1}{2}$
    % e.
    \item $P(\ket{\ibf}) = P(\ket{-\ibf}) = \tfrac{1}{2}$
    % f.
    \item From \ref{sol:2p2h} we know that $\ket{1} = -\tfrac{\ibf}{\sqrt{2}} (\ket{\ibf} - \ket{-\ibf})$.
    Therefore, $P(\ket{\ibf}) = P(\ket{-\ibf}) = \tfrac{1}{2}$.
    % g.
    \item It is useful here to perform a change of basis in order to express $\ket{+}$ in terms of its paired basis.
    To do this, we need to solve a set of linear equations.
    Writing out $\ket{+}$ in terms of the standard basis, we have,
    \begin{align*}
        \ket{+} = \spket{}{0}{+}{1}.
    \end{align*}
    We need to determined the components of $\ket{+}$ relative to the basis in which we are performing the measurement, requiring that,
    \begin{align*}
        \tfrac{1}{\sqrt{2}} &= c_1 \tfrac{1}{2} + c_2 \tfrac{\sqrt{3}}{2} \\
        \tfrac{1}{\sqrt{2}} &= c_1 \tfrac{\sqrt{3}}{2} - c_2 \tfrac{1}{2}.
    \end{align*}
    Solving gives,
    \begin{align*}
        c_1 &= \tfrac{\sqrt{2}}{4} (1 + \sqrt{3}) \\
        c_2 &= \tfrac{\sqrt{2}}{4} (1 - \sqrt{3}),
    \end{align*}
    leading to the probabilities,
    \begin{align*}
        P(\ket{a}) &= \tfrac{1}{4} (2 + \sqrt{3}) \\
        P(\ket{a^\perp}) &= \tfrac{1}{4} (2 - \sqrt{3}),
    \end{align*}
    where we have used $\{ \ket{a}, \ket{a^\perp} \}$ as a shorthand for the basis.
\end{enumerate}
